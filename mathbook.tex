%%%%%%%%%%%%%%%%%%%%%%%%%%%%%%%%%%%%%%%%%
% The MathBook
% LaTeX Template
% Version 1.0 (2024/06/XX)
%
% Based on main.tex and mathbook.cls
% Author: Your Name
% License: CC BY-NC-SA 4.0
%%%%%%%%%%%%%%%%%%%%%%%%%%%%%%%%%%%%%%%%%

\documentclass[
    11pt,
    fleqn,
    a4paper
]{mathbook}
\usepackage{kotex}

\begin{document}

% Book information for PDF metadata
\hypersetup{
    pdftitle={공통수학},
    pdfauthor={장규만},
    pdfsubject={Mathematics},
    pdfkeywords={Mathematics, Theorem, Proof, Example},
    pdfcreator={LaTeX},
}

%----------------------------------------------------------------------------------------
%   TITLE PAGE
%----------------------------------------------------------------------------------------

\begin{titlepage}
    \centering
    {\Huge\bfseries 공통수학\par}
    \vspace{16pt}
    {\LARGE A Modern Introduction\par}
    \vspace{24pt}
    {\huge\bfseries 장규만\par}
    \vfill
    {\large \today\par}
\end{titlepage}

%----------------------------------------------------------------------------------------
%   COPYRIGHT PAGE
%----------------------------------------------------------------------------------------

\thispagestyle{empty}
~\vfill
\noindent Copyright \copyright\ 2024 장규만\\
\noindent \textsc{Published by UTOEDU}\\
\noindent \url{https://www.example.com}\\
\noindent Licensed under the Creative Commons Attribution-NonCommercial 4.0 License.\\
\noindent \textit{First printing, June 2024}

%----------------------------------------------------------------------------------------
%   TABLE OF CONTENTS
%----------------------------------------------------------------------------------------

\pagestyle{empty}
\tableofcontents
\listoffigures
\listoftables
\pagestyle{fancy}
\cleardoublepage

%----------------------------------------------------------------------------------------
%   PART
%----------------------------------------------------------------------------------------

\part{Preliminaries}

%----------------------------------------------------------------------------------------
%   SAMPLE CHAPTER
%----------------------------------------------------------------------------------------

\chapter{Numbers and Sets}

\section{Natural Numbers}
The set of natural numbers is denoted by $\mathbb{N} = \{0, 1, 2, \ldots\}$.

\begin{theorem}[Well-Ordering Principle]
Every non-empty subset of $\mathbb{N}$ has a least element.
\end{theorem}

\begin{proof}
Suppose $S \subseteq \mathbb{N}$ is non-empty. ... (proof omitted)
\end{proof}

\begin{example}
Let $S = \{n \in \mathbb{N} : n \geq 5\}$. The least element is $5$.
\end{example}

\section{Integers and Rational Numbers}
The set of integers is $\mathbb{Z}$, and the set of rational numbers is $\mathbb{Q}$.

\begin{definition}[Rational Number]
A number $q$ is rational if $q = \frac{a}{b}$ for some $a \in \mathbb{Z}$, $b \in \mathbb{N}$, $b \neq 0$.
\end{definition}

\section{Exercises}
\begin{exercise}
Prove that the sum of two even numbers is even.
\end{exercise}

%----------------------------------------------------------------------------------------
%   ANOTHER CHAPTER
%----------------------------------------------------------------------------------------

\chapter{Functions and Relations}

\section{Functions}
A function $f$ from $A$ to $B$ is a rule that assigns to each $a \in A$ exactly one $b \in B$.

\begin{theorem}
If $f: \mathbb{R} \to \mathbb{R}$ is continuous and $f(0) > 0$, then there exists $\varepsilon > 0$ such that $f(x) > 0$ for all $|x| < \varepsilon$.
\end{theorem}

\section{Examples}
\begin{example}
Let $f(x) = x^2 + 1$. Then $f(0) = 1 > 0$.
\end{example}

%----------------------------------------------------------------------------------------
%   APPENDICES
%----------------------------------------------------------------------------------------

\begin{appendices}
\chapter{Additional Proofs}
\section{Proof of the Well-Ordering Principle}
(Proof details...)
\end{appendices}

\end{document} 