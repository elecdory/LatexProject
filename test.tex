\documentclass[a4paper,11pt]{book}

% 기본 패키지
\usepackage{amsmath,amssymb,amsthm}  % 수학 기호 및 환경
\usepackage{graphicx}                % 이미지 삽입
\usepackage{xcolor}                  % 색상 사용
\usepackage{enumitem}                % 리스트 사용자 정의
\usepackage{tcolorbox}               % 색상 박스
\usepackage{tikz}                    % 그래프 및 도형 그리기
\usepackage{hyperref}                % 하이퍼링크
\usepackage{fancyhdr}                % 헤더/푸터 사용자 정의
\usepackage{titlesec}                % 제목 형식 사용자 정의
\usepackage{multicol}                % 다중 열 사용
\usepackage{mathtools}               % 수학 도구 추가
\usepackage{kotex}                   % 한글 지원

% 여백 설정
\usepackage[left=2cm,right=2cm,top=2.5cm,bottom=2.5cm]{geometry}

% 페이지 헤더/푸터 설정
\pagestyle{fancy}
\fancyhf{}
\fancyhead[LE,RO]{\thepage}
\fancyhead[RE]{\leftmark}
\fancyhead[LO]{\rightmark}

% 사용자 정의 명령어
\newcommand{\levelone}{\textbf{Level 1}}
\newcommand{\leveltwo}{\textbf{Level 2}}
\newcommand{\levelthree}{\textbf{Level 3}}

% 문제 난이도 표시 명령어
\newcommand{\hard}{\textsf{상}}
\newcommand{\medium}{\textsf{중}}
\newcommand{\easy}{\textsf{하}}

% 문제 환경 정의
\newcounter{problemnumber}[section]
\newenvironment{problem}[1][]{%
	\refstepcounter{problemnumber}%
	\ifx&#1&%
	\par\noindent\textbf{\theproblemnumber}%
	\else%
	\par\noindent\textbf{\theproblemnumber} \textsf{#1}%
	\fi%
	\quad%
}{\par}

% 대표 문제 환경 정의
\newcounter{keycounter}[section]
\newenvironment{keyproblem}{%
	\refstepcounter{keycounter}%
	\par\noindent\textbf{대표 문제 \thekeycounter} \textsf{다시 보기}%
	\quad%
}{\par}

% 유형 환경 정의
\newcounter{typecounter}[chapter]
\newenvironment{problemtype}[1]{%
	\refstepcounter{typecounter}%
	\section*{유형 \thetypecounter \quad #1}%
}{\par}

% 정답과 해설 환경
\newenvironment{solution}{%
	\par\noindent\textbf{정답과 해설}%
	\quad%
}{\par}

\begin{document}
	
	% 표지 및 목차
	\title{\Huge{\textbf{수학 문제집}}}
	\author{저자 이름}
	\date{\today}
	\maketitle
	\tableofcontents
	\newpage
	
	% 머리말
	\chapter*{머리말}
	이 교재는 핵심 문제 중심으로 실속있게 공부할 수 있도록 구성되었습니다. 
	나에게 필요 없는 수준의 문제는 NO, 핵심만을 모은 군더더기 없는 구성으로 학습 효과 UP!
	연산 문제 중심으로 기본기를 확실하게 다지고, 단순 반복적인 연산 문제는 NO, 
	연산 유형을 체계적으로 구성하여 기초력 강화 UP!
	
	\newpage
	
	% 교재 구성
	\chapter*{교재의 구성}
	\begin{center}
		\begin{tikzpicture}
			\node[draw, rounded corners, fill=gray!20, minimum width=8cm, minimum height=2cm] 
			at (0,0) {\LARGE \textbf{Pattern Master}};
			\node[draw, rounded corners, fill=gray!20, minimum width=8cm, minimum height=2cm] 
			at (0,-3) {\LARGE \textbf{Arithmetic Master}};
		\end{tikzpicture}
	\end{center}
	
	% 목차
	\chapter*{차례}
	\begin{multicols}{2}
		\textbf{I. 다항식}\\
		01 다항식의 연산 \dotfill 8\\
		02 나머지정리와 인수분해 \dotfill 24\\
		03 복소수 \dotfill 48\\
		
		\textbf{II. 방정식과 부등식}\\
		04 이차방정식 \dotfill 62\\
		05 이차방정식과 이차함수 \dotfill 80\\
		06 여러 가지 방정식 \dotfill 94\\
		07 일차부등식 \dotfill 114\\
		08 이차부등식 \dotfill 128\\
		
		\textbf{III. 도형의 방정식}\\
		09 평면좌표 \dotfill 148\\
		10 직선의 방정식 \dotfill 162\\
		11 원의 방정식 \dotfill 182\\
		12 도형의 이동 \dotfill 202\\
	\end{multicols}
	
	% 본문 시작
	\chapter{다항식}
	
	\section{다항식의 연산}
	
	\begin{problemtype}{다항식의 덧셈과 뺄셈}
		다항식의 덧셈과 뺄셈은 괄호가 있는 경우 괄호를 푼 후 동류항끼리 모아서 간단히 한다.
		
		\begin{keyproblem}
			두 다항식
			$A=x^2-2xy+3y^2, B=3x^2+xy-y^2$
			에 대하여 2X-A=3A-2B를 만족하는 다항식 X를 구하여라.
		\end{keyproblem}
		
		\begin{problem}[{\easy\;\medium\;\hard}]
			세 다항식
			$A=2x^3-3x+4, B=-3x^2+2x, C=2x^3-x^2+1$
			에 대하여 2A-B-3(A-C)를 계산하면?
			\begin{enumerate}
				\item $-8x^3+6x^2+x-7$
				\item $-8x^3+x-1$
				\item $4x^3+6x^2+x-1$
				\item $4x^3+x-7$
				\item $4x^3+x-1$
			\end{enumerate}
		\end{problem}
		
		\begin{problem}[{\easy\;\medium\;\hard}]
			두 다항식 P, Q에 대하여 < P, Q >를
			< P, Q >=2P-Q+3
			이라고 할 때, < x+2y-1, 3x-4y+1 >을 계산하면?
			\begin{enumerate}
				\item $-x+3y$
				\item $-x+8y$
				\item $-x+8y+1$
				\item $5x-2y$
				\item $5x-2y+3$
			\end{enumerate}
		\end{problem}
		
	\end{problemtype}
	
	\begin{problemtype}{다항식의 전개식에서 계수}
		다항식의 전개식에서 특정 항의 계수를 구할 때는 분배법칙을 이용하여 필요한 항이 나오도록 계산한다.
		
		예) $(x^3+2x^2-x+1)(3x-1)$에서 $x^2$항은 $2x^2 \times (-1) + (-x) \times 3x = -5x^2$
		따라서 $x^2$의 계수는 $-5$이다.
		
		\begin{keyproblem}
			다항식 $(x^2+3x-2)(2x^2-x+6)$의 전개식에서 $x^2$의 계수는?
			\begin{enumerate}
				\item $-3$
				\item $-2$
				\item $-1$
				\item $1$
				\item $2$
			\end{enumerate}
		\end{keyproblem}
		
		\begin{problem}[{\easy\;\medium\;\hard}]
			다항식 $(x-2y-3)(4x+5y-6)$의 전개식에서 $xy$의 계수는?
			\begin{enumerate}
				\item $-3$
				\item $-2$
				\item $-1$
				\item $1$
				\item $2$
			\end{enumerate}
		\end{problem}
	\end{problemtype}
	
	% 최종 점검하기 섹션
	\section{최종 점검하기}
	
	\begin{problem}
		다항식 $(x+y+z)(x-y-z)$를 전개하여라.
	\end{problem}
	
	\begin{problem}
		다항식 $(x-4y)(x-2y)(x-y)(x+y)$를 전개하면?
		\begin{enumerate}
			\item $x^4-15x^3y+7x^2y^2-6xy^3-8y^4$
			\item $x^4-15x^3y+9x^2y^2+6xy^3-8y^4$
			\item $x^4-6x^3y-11x^2y^2+6xy^3-8y^4$
			\item $x^4-6x^3y+7x^2y^2+6xy^3-8y^4$
			\item $x^4+6x^3y-11x^2y^2-6xy^3-8y^4$
		\end{enumerate}
	\end{problem}
	
	\begin{solution}
		1. $x^2 - y^2 - z^2$
		
		2. 정답: (4) $x^4-6x^3y+7x^2y^2+6xy^3-8y^4$
	\end{solution}
	
	\chapter{방정식과 부등식}
	
	\section{이차방정식}
	
	\begin{problemtype}{이차방정식의 풀이}
		인수분해 또는 근의 공식을 이용한다.
		
		(1) $x$에 대한 이차방정식 $(ax-b)(cx-d)=0$의 근은 $x=\frac{b}{a}$ 또는 $x=\frac{d}{c}$
		
		(2) 계수가 실수인 이차방정식 $ax^2+bx+c=0$의 근은 $x=\frac{-b\pm\sqrt{b^2-4ac}}{2a}$
		
		참고) 계수가 실수인 이차방정식 $ax^2+2b'x+c=0$의 근은 $x=\frac{-b'\pm\sqrt{b'^2-ac}}{a}$
		
		\begin{keyproblem}
			이차방정식 $x^2+4x+2=0$의 해가 $x=\frac{a-\sqrt{b}}{2}$일 때, 유리수 $a, b$에 대하여 $a+b$의 값은?
			\begin{enumerate}
				\item $-2$
				\item $-1$
				\item $0$
				\item $1$
				\item $2$
			\end{enumerate}
		\end{keyproblem}
		
		\begin{problem}[{\easy\;\medium\;\hard}]
			이차방정식 $x(x+3)=3(x^2-1)-2x$의 해는?
			\begin{enumerate}
				\item $x=-\frac{1}{2}$ 또는 $x=-3$
				\item $x=-\frac{1}{2}$ 또는 $x=3$
				\item $x=-\frac{1}{3}$ 또는 $x=-1$
				\item $x=-\frac{1}{3}$ 또는 $x=1$
				\item $x=\frac{1}{2}$ 또는 $x=-3$
			\end{enumerate}
		\end{problem}
	\end{problemtype}
	
	% 이하 생략...
	
\end{document}